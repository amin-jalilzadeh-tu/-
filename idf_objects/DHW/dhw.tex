**Developing a Domestic Hot Water (DHW) Methodology Based on NTA 8800**  
*(circa 800 words)*

Designing an accurate methodology to calculate Domestic Hot Water (DHW) demand and its associated energy use is crucial for building performance simulations. The Dutch standard NTA 8800 (2024) provides a detailed framework for characterizing DHW needs, distribution losses, and generation efficiencies. Below, we describe how we extracted key concepts and formulas from NTA 8800, then translated them into a Python-based code structure for use in building energy models such as EnergyPlus.

---

## 1. Net Heat Requirement for Hot Tap Water

A starting point in NTA 8800 is determining the net heat requirement for hot tap water, \( Q_{W;nd} \). This value represents how much thermal energy the occupants need at the tap—before distribution or generation losses. NTA 8800 distinguishes between **residential** and **utility** buildings:

1. **Residential**  
   Section 13.2.2.1 outlines formulas that relate the number of occupants to the dwelling’s usable floor area. These occupant formulas approximate how many people live in a home of a given size. For instance, if \(A_{g, \text{zone}}\) is the floor area of the dwelling, NTA 8800 prescribes a piecewise approach that estimates occupant_count as a function of \(A_{g, \text{zone}}\). Once occupant_count is known, the standard sets a typical daily energy demand (in kWh) or water draw (in liters) per occupant.

2. **Utility Buildings**  
   For non-residential or “utility” functions, Section 13.2.2.2 gives a specific net heat demand per \(\text{m}^2\). For example, offices may be assigned around 1.4 kWh/m²/year, while healthcare or sports functions have higher values. The formula often looks like:
   \[
   Q_{W;nd;z,mi} = Q_{W;nd;spec;us} \times A_{g;z}
   \]
   where \( Q_{W;nd;spec;us} \) is the usage-function-specific demand (kWh/m²/year), and \( A_{g;z} \) is the zone’s usable floor area.

In our Python code (`parameters.py`), we combined these approaches by either computing daily liters from occupant_count × liters_per_person_per_day (for residential) or from area × (liters_per_m2_per_day) (for utility). We also allow the user to override these ranges if custom data are available.

---

## 2. Distribution and Delivery Efficiencies

NTA 8800 splits piping heat losses into two main categories:

- **Delivery efficiency**, \(\eta_{W;em}\), capturing near-tap losses such as draw-off pipes and the time it takes for hot water to reach the tap.  
- **Distribution efficiency**, \(\eta_{W;dis}\), capturing heat losses in circulation loops or long distribution lines.

For **residential** buildings, Section 13.3.3.1 provides tables that link pipe length to an efficiency factor. An example is a short pipe length of 2 m might yield \(\eta_{W;em} = 1.00\), whereas longer runs (8 m or more) reduce efficiency substantially. For **utility** buildings, Section 13.3.3.2 lumps this into a simpler table, typically giving 1.0 for very short average lengths or lower for longer piping.

In our Python workflow, we considered these losses optionally:  
1. **If** the user wants a simple model, we default to “no piping losses” (or an \(\eta_{W;dis} = 1.0\)).  
2. **If** a user supplies piping lengths and circulation details, we incorporate a distribution loss factor.

---

## 3. Storage Tank Standby Losses

Section 13.6 of NTA 8800 details how hot water storage tanks lose heat to the surroundings, typically referred to as standby losses. The formula often used is:

\[
Q_{W;sto;ls} = H_{sto;ls} \times \bigl(T_{\text{set}} - T_{\text{amb}}\bigr) \times \frac{t_{\text{period}}}{1000}
\]

where \(H_{sto;ls}\) (in W/K) is the tank’s heat transfer coefficient, \(T_{\text{set}}\) is the hot water setpoint (often 60 °C), \(T_{\text{amb}}\) is the ambient temperature, and \(t_{\text{period}}\) is the number of hours per month or year. NTA 8800 also includes “label-based” or “manufacturer-based” standby values (e.g., for small electric boilers).

In the Python code, we introduced a parameter `tank_standby_loss_w` that can be set either from a default table or from the user’s known data. Then, at runtime, we convert that to monthly or daily kWh losses and add it to the final hot water energy consumption.

---

## 4. Generation Efficiency

NTA 8800 requires factoring in the efficiency of water heaters, boilers, heat pumps, or even district heat substations. For example:

- **Electric boilers** often assume an efficiency of ~1.0 at the building boundary (though a primary energy factor for electricity applies elsewhere).  
- **Condensing gas boilers** can be around 0.9 if tested in standard conditions.  
- **District heat** is assigned a “generation efficiency” of 1.0 at the building boundary, plus a separate primary energy factor in the final building performance rating.

If we have a known device type, we store an approximate `device_efficiency` in `assign_dhw_parameters()`. For instance, a combi gas boiler might get 0.90. Then, in the final energy calculation, we divide net demand by that efficiency to see how much fuel or electricity is required.

---

## 5. Shower Water Heat Recovery (Optional)

Section 13.5 describes “Douche WTW” (DWTW), or shower water heat recovery, which can reduce hot tap water demand. The standard uses a factor to represent how much heat from the shower drain can be recovered. We have an optional “shower_recovery_factor” in our code if a building includes DWTW. If present, we reduce the occupant’s daily liters or the net heat requirement accordingly.

---

## 6. Implementation in Python

- **Mapping Building Types:** `map_building_function_to_dhw_key()` uses building function (e.g., “residential_sfh_small” vs. “Office”) to choose relevant parameter ranges and occupant formulas.  
- **Assigning DHW Values:** `assign_dhw_parameters()` merges default references (which can be NTA-based or user-based) and picks a final set of parameters such as `liters_per_person_per_day`, `tank_volume`, or `device_efficiency`.  
- **Parameter Calculations:** `calculate_dhw_parameters()` uses occupant_count or area-based calculations from the chosen formula, optionally adjusts for distribution efficiency or standby loss, and arrives at final numeric results (daily liters, peak flow rates, heater capacity, etc.).  
- **Schedules & WaterHeater:** In `create_dhw_schedules()`, we produce daily usage fraction schedules. Finally, `add_dhw_to_idf()` injects these parameters into an EnergyPlus or IDF model.

Through these steps, we ensure our Python code aligns with the major elements of NTA 8800: net heat requirement, distribution/delivery losses, storage tank standby losses, and generation efficiency. While not all NTA 8800 details (like complex monthly partitioning or fully recoverable losses in large buildings) are mandatory for every project, this structured approach enables us to add more fidelity if needed. Ultimately, we provide a flexible yet standardized way to estimate DHW energy use in compliance with Dutch regulations.





==========================================================
==========================================================
==========================================================

We began our endeavor by exploring how to align our domestic hot water (DHW) usage modeling with the methodology proposed by NTA 8800. This standard, widely used in the Netherlands for building energy performance, provides both occupant‐based and area‐based approaches for computing the net heat demand associated with hot tap water. In order to incorporate these calculations into our Python code, we needed to move beyond a simple dictionary of static values. Instead, we relied on the occupant or usage formulas described in NTA 8800, particularly in Sections 13.2 and 13.2.3, to determine occupant counts and daily liters demand for each building type. By following these formulas, we ensure that the occupant‐related data in our model reflect typical and officially recognized values, thereby making our simulation outputs significantly more robust and standardized.

One of the essential components of NTA 8800 is the differentiation between residential and nonresidential usage. For residential buildings, the standard prescribes formulas that link the usable floor area of a dwelling to an occupant count. An example of a simplified approach is: occupant_count = 1 if area ≤ 50, occupant_count = 1 + 0.01 × (area − 50) if area > 50, and so forth. Though this is only a condensed illustration of the real piecewise logic in NTA 8800, it shows how occupant counts become systematically tied to the dwelling’s size. In actual practice, the standard includes more refined formulas—for instance, occupant counts can increase incrementally across multiple size thresholds. Once we compute occupant_count from such a formula, we can derive a daily liter consumption based on occupant_count × 40–50 liters per day, a range that NTA 8800 commonly associates with typical Dutch residential dwellings at around 60 °C supply temperature. Mathematically, if occupant_count = N, we might say daily_liters = N × 45, meaning that, for a midsized home with occupant_count = 2.5, daily_liters would be 112.5 liters per day.

In contrast, for nonresidential buildings, NTA 8800 often uses net heat demand in kWh per square meter per year, as described in Table 13.1. That table assigns different figures to typical building functions, such as 1.4 kWh/m²·year for an office function, 2.8 kWh/m²·year for a meeting function, or around 15.3 kWh/m²·year for healthcare with beds. The code must then convert this net heat demand into daily liters at 60 °C. If we call Qannual (kWh per year) the net hot‐water heat demand, the standard suggests that 1 kWh is approximately equivalent to 13.76 liters of water heated by 50 K. Consequently, daily_liters = (Qannual × 13.76) / 365. For instance, an office of 100 m² at 1.4 kWh/m²·year would yield Qannual = 1.4 × 100 = 140 kWh. In liters, that would be 140 × 13.76 ≈ 1926 liters per year, or around 5.28 liters per day. One can then further allocate that daily usage across a certain occupant count or directly treat it as an aggregated figure if occupant counts are irrelevant for the nonres building’s hot water distribution. Formally, if occupant_density is given, occupant_count = floor_area / occupant_density, and liters_per_person_per_day = daily_liters / occupant_count, else occupant_count might default to 1, meaning the entire daily_liters is attributed to a single occupant.

In our Python code, this logic unfolds inside our function assign_dhw_parameters. That function begins by identifying the building function—residential or nonresidential—and then determines whether to use occupant‐based or area‐based usage. If building_function == "residential," we apply the occupant formula from NTA 8800, typically occupant_count = 1 for very small dwellings or occupant_count = 1 + 0.01 × (area − 50) for mediumsized ones, though these numeric coefficients are placeholders for demonstration. Once occupant_count is computed, daily_liters is occupant_count × 45, and we recast that into liters_per_person_per_day by dividing daily_liters by occupant_count. In so doing, we override whatever occupant_density or liters_per_person_per_day might have been initially assigned in a dictionary of fallback values. Then, in the final dictionary we return, occupant_density becomes None, reflecting that we used a direct occupant count approach. This ensures that all subsequent calculations, such as schedule shape factors, remain consistent with the occupant usage from the standard. 

Meanwhile, if building_function is "non_residential," we refer to building_row["non_residential_type"] and consult the dictionary of net heat demand factors from Table 13.1. Suppose that dictionary indicates 2.8 kWh/m²·year for "Meeting Function" or 1.4 kWh/m²·year for "Office Function." Multiplying that factor by the building’s area yields an annual net hot‐water heat demand in kWh, which we convert to liters by Qannual × 13.76 / 365. If occupant_density is available, we proceed to occupant_count = area / occupant_density. Then liters_per_person_per_day = daily_liters / occupant_count. If occupant_density is missing, we instead define occupant_count = 1 and occupant_density = area, so that occupant_count × occupant_density = area, effectively meaning the entire building usage lumps into one occupant or no occupant distribution is needed. This approach remains consistent with how one might handle usage in the standard, albeit in a simplified manner.

We used the code structure to store these final occupant_density and liters_per_person_per_day inside a dictionary called result. That same dictionary also holds default_tank_volume_liters, default_heater_capacity_w, setpoint_c, usage_split_factor, peak_hours, sched_morning, sched_peak, sched_afternoon, and sched_evening. Each of these parameters can come from param ranges we define in the dhw_lookup. In NTA 8800, setpoint temperatures might be around 60 °C for normal conditions or 65 °C if recirculation or certain healthcare contexts apply, so we keep a range such as setpoint_c_range = (58.0, 62.0) in the dictionary. Then we either pick the midpoint or a random value from that range. The usage_split_factor and sched_xxx parameters are not spelled out in the standard but help shape the distribution of daily usage across different times of day. By storing all these in the final result dict, we can pass them to a logging dictionary assigned_dhw_log, thereby making it possible to trace how occupant usage or daily liters values were derived.

In effect, this logic unifies occupant or net heat demand methods from NTA 8800 with the param ranges and fallback definitions in our dhw_lookup. Instead of referencing an older dictionary called BUILDING_TYPE_DHW_DATA, we rely solely on these param ranges, plus the occupant usage overrides from the standard. Code calls the function assign_dhw_parameters to retrieve occupant densities or daily liters, optionally employing the newly introduced use_nta flag to enforce the official occupant or net demand formula from NTA. That process is fundamental to our alignment with NTA 8800. We thus ensure that each building’s occupant usage or area usage is derived in a manner that reflects both standard practice and local calibration. When building_function is "residential," occupant_count is formula-based; when it is "non_residential," net heat demand is drawn from Table 13.1. The result is a single interface for code that wants the final occupant_density or liters_per_person_per_day, yet we have a robust underlying logic that respects NTA occupant or usage assumptions.

The final step in the chain occurs when we pass assigned parameters and occupant counts into a scheduling or water heater creation method. By that point, we have occupant_count, daily liters, and a schedule shape that we can combine in any form needed for energy simulation software. The advanced distribution of the usage can also be shaped further by referencing NTA 8800’s Annex T for a more detailed 24‐hour load profile, but we have decided for now to maintain a simpler fraction-of-peak approach. This interplay of NTA occupant usage formula, random or midpoint param ranges, and external overrides ensures we can either do an approximate alignment for initial design or a strict alignment if our user sets use_nta=True. Our final dictionary of occupant_density, liters_per_person_per_day, setpoint, and so forth thus stands as a cohesive snapshot of the building’s hot water assumptions, fully traceable to the standard and to the user’s own calibration needs.









========================================================

========================================================

========================================================







Below is a **concise guide** on **which parts of NTA 8800** are **most relevant** for extracting **DHW (Domestic Hot Water) parameters** and **how** you can weave them into your Python/EnergyPlus workflow—so that the final code (like you’ve shown) produces **realistic, compliance-oriented** values. 

---

# 1. Key NTA 8800 Sections and What You Extract

1. **\S13.2.2.1 – Determining Number of Occupants**  
   - **Why**: In residential cases, NTA 8800 uses occupant‐based net heat demand for DHW. There are **formulas** (e.g., 13.16–13.18) that define occupant count \( N_{p} \) from building area \( A_{g} \).  
   - **How to Use**: In your code, if the building function is *residential*, you can apply these occupant formulas when `occupant_count` is unknown. If `occupant_count` **is** known, you skip the formula.

2. **\S13.2.3.1 (and Annex T) – Daily Hot Water Volume per Person**  
   - **Why**: This section gives typical **liters/person/day** at 60 °C (~40 L/day/person in many Dutch references).  
   - **How to Use**: Populate your `liters_per_person_per_day` in the dictionary (for example, 40–50 L/day). That way your code’s occupant-based formula automatically picks up these values.

3. **\S13.6.2–13.6.3 – Storage Tank Standby Losses**  
   - **Why**: If you need more accurate **tank loss coefficients** (W/K), NTA 8800 provides guidance on how to handle or estimate them.  
   - **How to Use**: In your `WaterHeater:Mixed`, you have fields like `Off_Cycle_Loss_Coefficient_to_Ambient_Temperature`. You could tie those to the NTA standby formulas or typical “lumped” W/K values recommended in that subsection.

4. **\S13.8 – Generation Device Efficiencies**  
   - **Why**: Whether it’s a condensing gas boiler or electric immersion heater, NTA 8800 lists typical (or default) efficiencies.  
   - **How to Use**: In your code, set `Heater_Thermal_Efficiency` accordingly—e.g., `0.90` for a decent gas boiler, `1.0` for a simple electric heater, or up to `0.95` if condensing, etc.

5. **\S13.1.1.1 and \S13.3.2 – Setpoint Temperatures**  
   - **Why**: Commonly states around **60 °C** for typical DHW, sometimes **65 °C** if there’s a circulation loop or legionella control.  
   - **How to Use**: This matches your code’s default `setpoint_c=60.0`; you can allow a user config or logic to bump to 65 °C if needed.

6. **Annex T – Detailed 24‐hr Tapping Patterns**  
   - **Why**: If you ever want **hourly** draw patterns (like morning/evening spikes) as recommended for advanced compliance checks or shower calculations.  
   - **How to Use**: Your code’s fraction schedules can reflect these detailed “Annex T” patterns if you want more realism, rather than a simple “peak hours” approach.

---

# 2. Practical Steps for Incorporating NTA 8800 Values

Below is how you’d **plug** these NTA 8800 references into your existing **Python code**.

1. **Occupant Count**  
   - If `building_function` is “residential” and occupant_count is **missing**, implement the formula from \S13.2.2.1:  
     \[
       N_{p} = 
       \begin{cases}
         1.28 + 0.01 \times \frac{A_{g}}{N_{\text{woon}}}, & \text{(for certain area ranges)}\\
         \quad \dots & \dots
       \end{cases}
     \]  
   - Or use the simpler default: ~\( \frac{A_{\text{floor}}}{30} \) if you prefer a quick occupant guess.  
   - Then pass this \(N_{p}\) to your `assign_dhw_parameters` or `calculate_dhw_parameters` function.

2. **Liters/Person/Day**  
   - Use the ~**40 L/day** at 60 °C from \S13.2.3.1 or the exact table (some references say 40.29 L/day).  
   - In your `dhw_lookup` or `BUILDING_TYPE_DHW_DATA` dictionary, set `liters_per_person_per_day = 40.0`.  
   - In **nonresidential** contexts, if you want an occupant-based approach, you can cross-check \S13.2.2.1 or 13.2.3.1. (NTA is mostly residential, but you may adapt the occupant logic similarly.)

3. **Tank Volume**  
   - NTA doesn’t strictly fix tank volumes but you can do a “full-day storage” approach:  
     \[
       V_{\text{tank\_liters}} \approx N_{p} \times 40 \times \text{(days of storage fraction)}
     \]  
   - Or rely on NTA’s method for daily kWh and do the ~17 L/kWh conversion ( \S13.2.2.1, plus typical 4.19 kJ/kgK for water, 50 K delta, etc.).  
   - Insert that final number into your code’s dictionary or pass via `user_config_dhw`.

4. **Heater Capacity (W)**  
   - Per \S13.8, choose an **efficiency** (gas ~0.90, electric ~1.0).  
   - Then from your daily kWh or the desired “recovery time,” compute the size:  
     \[
       P_{\text{heater}} = 
         \frac{Q_{\text{daily}}(\text{kWh})}{\text{recovery\_hours}} 
         \times 1000 \text{ (to get W)}.
     \]  
   - That goes directly into your `default_heater_capacity_w`.

5. **Setpoint**  
   - From \S13.1.1.1 or \S13.3.2, 60 °C is the standard. Possibly 65 °C if a circulation loop.  
   - In your code, that populates `setpoint_c`.

6. **Standby Loss Coefficients**  
   - \S13.6.2–3 offers a method to determine or approximate W/K. If you get that from, say, table in the annex or from a known test standard, feed it into:  
     ```python
     wh_obj.Off_Cycle_Loss_Coefficient_to_Ambient_Temperature = your_estimated_value
     wh_obj.On_Cycle_Loss_Coefficient_to_Ambient_Temperature  = your_estimated_value
     ```
   - Or set it in your code’s dictionary if you want a default (like 3 W/K).

7. **Schedules** (Annex T)  
   - If you want advanced occupant-based usage patterns from Annex T, define a more **granular** fraction schedule. For instance, large spike from 6–9 AM, smaller midday usage, another spike 7–9 PM.  
   - This can replace your simpler “peak_hours” approach in `assign_dhw_parameters` or `create_dhw_schedules`.

---

# 3. Why This Makes Your Code “Realistic”

1. **Official Formulas**: By using **NTA’s occupant formula** and **liters/day** references, you’re reflecting the actual Dutch building code assumptions.  
2. **Documented**: You can say “**We used occupant formulas from NTA 8800 \S13.2.2.1**” or “**We used 40 L/day from \S13.2.3.1**.” This helps with compliance documentation or audits.  
3. **Scalable**: If the occupant density changes or NTA updates the standard, you just tweak those parameters—no changes to the rest of your (well-structured) code.

---

# 4. “Quick” Cheat Sheet of Values from NTA 8800

- **Occupant Count**:  
  - \(\approx\frac{A_{g}}{30}\) as a rough fallback, or use (13.16–13.18) if you want the official formula.  
- **Liters per Person**: ~**40–45 L/day** at 60 °C (some references mention 40.29 L/day).  
- **Setpoint**: Typically **60 °C**; up to **65 °C** for anti‐legionella in circulation loops.  
- **Tank Standby**: \(\approx 2–4\,\text{W/K}\) if you have an average insulated tank, or see \S13.6.3 for more exact.  
- **Heater Efficiency**: ~**0.90** for a gas combi, **1.0** for direct electric, **0.95** for a condensing gas.  
- **Heater Capacity**: Freed by user preference or code. For small dwellings, typically 3–6 kW electric or ~20–30 kW gas.

---







- **Relevant NTA Sections**: \S13.2.2.1 (occupants), \S13.2.3.1 (daily liters), \S13.6.2–3 (tank losses), \S13.8 (generator efficiency), \S13.1.1.1 & \S13.3.2 (setpoints), Annex T (detailed tapping patterns).  
- **Plug In**: Use them to fill occupant counts, tank volumes, heater capacity, and schedule shapes.  
- **Result**: A more **compliant** and **traceable** occupant-based DHW model in EnergyPlus, exactly what NTA 8800 aims for.

That’s essentially **all** you need from NTA 8800 to make your existing DHW code produce **realistic, code-aligned** results for residential—and even partially for nonresidential—buildings.









Short NTA-based Rationale
Per NTA 8800 section 13.2, if the building is “residential,” we adopt occupant-based hot water demand. The standard occupant usage is about 40 L (at 60 °C) per occupant per day. For “utility,” we use typical “kWh/m² per year” or “liters at 60 °C per m²/day” from table 13.1. E.g., offices: 0.065 L/m²/day, sports: 0.588 L/m²/day, etc.



Short NTA-based Rationale
NTA 8800 typically expresses DHW demand as net heat demand in kWh. However, they assume around 40 L (at 60 °C) per occupant per day for dwellings, which you can keep in “liters_per_person_per_day” form. If needed, you can convert liters → kWh using standard water properties and temperature rise. For utilities, refer to Table 13.1 to define “kWh per m²/yr” or “liters per m²/day.”



Short NTA-based Rationale
NTA 8800 requires at least 60 °C for DHW to avoid Legionella risk. For circulation systems, often 65 °C is used so that the loop remains at or above 60 °C. Hence we pick 60 °C or 65 °C accordingly.



Short NTA-based Rationale
For small residential, a 200 L tank plus ~4 kW heater is consistent with typical Dutch single-family scenarios. NTA 8800 doesn’t fix a universal volume/capacity but references that single-family systems range ~80–200 L and ~3–6 kW. Larger dwellings or utilities might have 300+ L tanks and 10–20 kW. If you want to incorporate standby losses, see section 13.6 for the standard’s formula to compute storage heat losses based on setpoint, ambient temperature, and tank heat loss coefficient.




Below is a high-level *map* of which parts of the NTA 8800 text are **most relevant** to the code you’ve shown. I’ve grouped them by how they might align with your code structure (e.g. net heat requirement, distribution losses, storage tank losses, generation efficiency, etc.). You can decide which details to implement immediately versus which ones to keep as references. 

---

## 1. Net Heat Requirement for Hot Tap Water

**Relevant NTA 8800 Sections**  
- 13.1 (overall principle of hot water energy use)  
- 13.2 (net heat requirement and tapping patterns)  
- 13.3.3.1 / 13.3.3.2 (delivery efficiencies for residential/utility)  
- 13.5 (shower water heat recovery, if you want to reduce net demand)  

**How It Ties Into Your Code**  
1. In your code, the function `calculate_dhw_parameters()` ultimately produces `daily_liters` or daily hot water volume.  
2. NTA 8800 Chapter 13 determines the net heat requirement (\(Q_{W;nd}\)) from occupant counts (residential) or from a specified kWh/m² (utility).  
3. If you want to be fully NTA 8800–compliant, you might incorporate these references:  
   - *Residential category* (13.2.2.1) => net heat demand for hot tap water is based on an average number of occupants that depends on floor area plus a standard usage (in kWh/person).  
   - *Utility category* (13.2.2.2) => net heat demand is based on usage function (office, healthcare, etc.) times a fixed kWh/m²/year.  

**Possible Implementation Ideas**  
- In your `assign_dhw_parameters()` or in `map_building_function_to_dhw_key()`, you already separate “residential” vs. “non_residential.” You could inject an *NTA-based formula* for occupant_count or daily demand:
  - For houses, NTA has a formula that yields occupant_count from area.  
  - For offices, it’s a standard 1.4 kWh/m²/year or so (Table 13.1 in your excerpt).  
- Then your “liters_per_person_per_day” or total daily liters can be replaced or cross-checked with the NTA’s values.  

---

## 2. Distribution Losses (Piping, Circulation, Etc.)

**Relevant NTA 8800 Sections**  
- 13.3 (delivery efficiency for the local “outlet pipes” or “tap lines”)  
- 13.4 (distribution losses)  
- 13.4.3 (circulation system heat losses, distribution efficiency \(\eta_{W;dis}\))  

**How It Ties Into Your Code**  
1. In your code, you have a `create_dhw_schedules()` which sets up fraction schedules, and you also define a `peak_flow_m3s` in `calculate_dhw_parameters()`.  
2. The NTA 8800 has a separate concept of:
   - **Delivery efficiency** (\(\eta_{W;em}\)): covers the losses *from the tap point* itself + small piping.  
   - **Distribution efficiency** (\(\eta_{W;dis}\)): covers the losses from any circulation loop or longer supply lines.  
3. If you want to incorporate distribution losses explicitly, NTA 8800 details how to:
   - Distinguish *small outlet pipe losses* vs. *longer distribution system or circulation loop.*  
   - Possibly treat the combined effect of these as an overall multiplier or separate them.  

**Possible Implementation Ideas**  
- You might add new code or new fields in `assign_dhw_parameters()` that store “distribution_efficiency” or “delivery_efficiency” if you want a separate factor from the default 1.0.  
- Or you can keep it simple by adjusting your “usage_split_factor” or “peak_hours” to incorporate typical distribution inefficiencies.  

---

## 3. Storage Tank (Standby Losses)

**Relevant NTA 8800 Sections**  
- 13.6 (storage losses)  
- 13.6.2 (calculation rules for storage tank losses)  

**How It Ties Into Your Code**  
1. Your code has `default_tank_volume_liters` and `heater_capacity_w`.  
2. NTA 8800 goes into detail about *standby losses* from storing hot water, especially in bigger tanks or boilers.  
3. If you want a more detailed approach, you could:
   - Use a simple constant “tank loss factor” or “standby loss factor” if you just want a single multiplier.  
   - Or do a full calculation from \(H_{sto;ls}\) or \(S_{sto;ls}\) (the W/K or kWh/day losses).  

**Possible Implementation Ideas**  
- You could store “storage_tank_standby_loss_w” as an additional key in `assign_dhw_parameters()`.  
- Then in `calculate_dhw_parameters()`, you could incorporate that standby loss into your final daily usage or daily energy.  
- Alternatively, if you only want a *rough approach*, you might add a flat X% loss to your daily hot water load.  

---

## 4. Generation Efficiency (Boilers / Heat Pumps / District Heat)

**Relevant NTA 8800 Sections**  
- 13.8 (heat generation) — big chunk  
- 13.8.4 (details on generation devices)  
- 13.8.4.9 (district heat, external supply = efficiency of 1.0 at building boundary)  

**How It Ties Into Your Code**  
1. Right now in `assign_dhw_parameters()`, you define a “default_heater_capacity_w” but no separate “efficiency.”  
2. In NTA 8800, the generation efficiency can come from: 
   - Measured data (if tested), or  
   - A table with typical “0.8”, “0.9,” etc., or  
   - “1.0” if it’s an electric water heater or external district heat supply (but plus the primary energy factor in final stage).  
3. If you want to reflect real device performance, you could store an “efficiency” or “cop” for a heat pump, or “0.9” for a gas condensing boiler.  

**Possible Implementation Ideas**  
- Add an optional “device_efficiency” or “device_type” in your parameters.  
- If “electric,” you can set `device_efficiency=1.0` (assuming no real stack losses). If “condensing gas boiler,” maybe 0.9.  
- Then feed that into your final energy calculation for “daily_liters × 4.186 × (temp delta) / efficiency.”  

---

## 5. Recoverable Losses / Gains

**Relevant NTA 8800 Sections**  
- 13.4.5 / 13.5.4 / 13.6.5 / 13.1.1.5 discuss *recoverable losses* that might partially heat the building in winter.  

**How It Ties Into Your Code**  
- If you model the building’s *space heating* plus *DHW* together in EnergyPlus or similar, sometimes NTA 8800 wants you to account that certain pipe losses might be helpful in winter.  
- Your code is not obviously including explicit “recoverable losses,” so you might omit that detail or handle it in the building energy model.  

---

## 6. Shower Water Heat Recovery (DWTW)

**Relevant NTA 8800 Sections**  
- 13.5 (DWTW)  

**How It Ties Into Your Code**  
- If you want to reduce the net hot water demand from showers specifically, you can incorporate a factor for “shower heat recovery.”  
- Your code does have an attribute `usage_split_factor`, but no dedicated logic for “DWTW.” You *could* add that logic if you want to reflect a “savings” on the water heating side.  

---

## Putting It All Together

Below is a rough outline of *how* you might incorporate or map the NTA 8800 logic into your existing Python functions:

1. **Map Building to DHW Key** (`building_type_map.py`)  
   - (Optional) Insert logic that says: if `building_function == 'residential'`, look up occupant counts using NTA 8800 formula. If `non_residential_type == 'Office'`, use the standard kWh/m² from table 13.1, etc.

2. **Assign DHW Parameters** (`assign_dhw_parameters.py`)  
   - *Enhancement 1*: inject occupant formulas from NTA if you want.  
   - *Enhancement 2*: add “distribution_efficiency” or “delivery_efficiency” keys if needed.  
   - *Enhancement 3*: add “tank_standby_loss” if you want more detail.  
   - *Enhancement 4*: add “device_efficiency” if you want to incorporate generation device performance from NTA tables.

3. **Calculate DHW** (`parameters.py`)  
   - Right now you do `occupant_count * liters_per_person_per_day` to get daily liters. If you want an NTA 8800–style approach, you can do one of:  
     1. Residential: occupant_count = function_of_area, daily_liters = occupant_count × typical liters.  
     2. Nonresidential: daily_liters = area × (liters/m²).  
   - *Add step for distribution losses or tank losses.*  
   - *Optionally adjust for device efficiency.*  
   - *Optionally incorporate DWTW (shower heat recovery) if you want advanced detail.*  

4. **Schedules** (`schedules.py`)  
   - The NTA’s monthly or daily distribution approach can be integrated if you want time-of-day or monthly shape. This is optional.

5. **Water Heater** (`water_heater.py`)  
   - (If you want to link to EnergyPlus) you might incorporate an additional “Heater Thermal Efficiency” from the NTA approach. Or for “District Heating,” you set Fuel Type = DistrictHeating with an efficiency of 1.0 at the boundary but different primary energy factors behind the scenes.  

---

## Summary

- **Most** NTA 8800 details revolve around *where* the net demand numbers come from ( occupant-based or area-based ) and *how* distribution, generation, and standby losses are factored in.  
- **Your code** can remain mostly the same structure, but you’d add or override certain default parameters with NTA-based logic ( occupant count formula, distribution losses, device efficiency, etc. ).  
- You **do not** have to implement every detail (recoverable losses, full monthly angle, PFHRD, partial usage classes, etc.) unless your project specifically requires full compliance.  

These highlights should help you see which chunks of the NTA 8800 text are relevant—and *how* they might flow into your existing Python code.






