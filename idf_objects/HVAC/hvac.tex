NTA 8800 itself does not give detailed separate day/night setpoint schedules for each building type. In practice, the standard typically uses a 20–21 °C day setpoint and 15–16 °C night setpoint across many building types. That said, you can expand on them to reflect real-world differences (if desired) or keep them nearly the same for all building functions.

   
In our work, we incorporated various elements of the Dutch energy performance standard NTA 8800 into a Python-based workflow that generates building HVAC parameters, occupant schedules, and setpoints for use in EnergyPlus. The core objective was to remain faithful to the assumptions and methodology recommended by NTA 8800, particularly for setpoints, schedules, and temperature corrections, while still allowing the flexibility of a simulation-based approach. Our script reads basic metadata from a dataset of buildings, combines this information with a nested lookup table (hvac_lookup), and can optionally apply user-configurable overrides from an Excel file. This ensures that each building receives appropriate temperature setpoints, schedule information, and supply air temperatures that reflect Dutch standards of occupant comfort and typical usage.

In NTA 8800, the energy demand for heating and cooling in a given month is determined by a set of formulas that account for heat transfer, ventilation, and system characteristics. Although our tool focuses primarily on generating day and night temperature setpoints for an Ideal Loads Air System, rather than modeling the distribution and generation efficiencies in detail, we still integrated many of the front-end assumptions from NTA 8800. For instance, we rely on typical occupant comfort values of around 20 °C during the daytime and 15 or 16 °C at night for residential spaces, or slightly higher day setpoints (perhaps 21 or 22 °C) for certain non-residential uses. For cooling, we adopt the principle that the occupied day setpoint might be around 24 to 26 °C, while nighttime can allow 26 to 28 °C when occupants are absent. These align with the standard practice in NTA 8800 that occupant comfort temperatures for heating typically fall near 19–21 °C, and cooling setpoints are in the mid-twenties.

Another important set of concepts in NTA 8800 concerns temperature corrections. Section 9.3.2 of the standard introduces the idea that a room’s effective temperature can deviate slightly from the thermostat setpoint, depending on whether the delivery system is based on radiators, floor heating, or air-based heating. For example, if a building employs underfloor heating, NTA 8800 might specify a correction factor of 0.3 K, which effectively means the system may need to heat the water or air slightly above the nominal occupant setpoint to compensate for distribution losses or a gentle ramp-up in the radiant floor. While our Python-based script does not strictly apply these fractional corrections to the final zone temperature in the Ideal Loads system, it can add them internally if desired, or store them in the final assigned log so that the reported “heating_day_setpoint” has accounted for the offset. Similarly, the standard recognizes that certain control systems, such as thermostatic radiator valves or advanced building management systems, can reduce or increase the net temperature offset by a small margin. If desired, we can reflect these control-based offsets as small numeric shifts in the final setpoints by referencing the corresponding table in section 9.3.2.

In addition to temperature setpoints, NTA 8800 also outlines how to handle schedules and occupancy. Our approach assigns a schedule_details sub-dictionary for each building in the hvac_lookup structure, containing entries such as day_start, day_end, and occupancy_weekday. For residential buildings, day_start might be 07:00 and day_end might be 23:00, indicating minimal setback at night, whereas for a non-residential “Meeting Function,” day_start could be 08:00 and day_end might be 22:00 on weekdays, with reduced hours on weekends. Because NTA 8800 lumps these patterns into a monthly calculation, it does not typically require a minute-by-minute schedule, but we found that storing these time blocks in a schedule_details field allows us to convert them to SCHEDULE:COMPACT objects in EnergyPlus if desired. The monthly method in NTA 8800 states that the required energy for heating, QH;nd,zi,mi, can be determined by summing the heat losses from transmission and ventilation, factoring in recoverable internal gains. Our script does not implement that monthly calculation itself but supplies occupant-driven and time-of-day-driven setpoints, which let EnergyPlus compute the loads in a more granular simulation. 

Where infiltration is concerned, NTA 8800 references different infiltration or air leakage assumptions depending on the age of construction. For example, a building from the period 1900–2000 might exhibit significantly higher infiltration rates than a more modern building from 2000–2024 unless proven otherwise with an airtightness test. Our workflow acknowledges this by letting the user store infiltration values in a separate module, keyed to the same building attributes: calibration_stage, scenario, building_function, subtype, and age_range. Although the formula for infiltration in NTA 8800 is fairly direct, we simply rely on typical nominal infiltration rates in the script. If the user needs occupant-based infiltration or a dynamic infiltration schedule, these can be loaded or set as part of a separate infiltration_lookup but keyed to the same building_row data.

On the coding side, we maintain a structured Python dictionary hvac_lookup that distinguishes calibration_stage, scenario, building_function, subtype, and age_range. Each final sub-dict contains keys such as heating_day_setpoint_range, max_heating_supply_air_temp_range, and schedule_details. Our assign_hvac_ideal_parameters function receives a building_row with fields like building_function, residential_type, non_residential_type, age_range, and scenario, and navigates the multi-level dictionary to retrieve the relevant setpoint ranges. If a user wants to override these defaults, they can supply an Excel sheet with columns for calibration_stage, scenario, building_function, param_name, min_val, max_val, fixed_value, and optionally schedule_key or schedule_value. Each row in that sheet is read into a nested override_data dictionary that mirrors the structure of hvac_lookup. For formula-based references, we handle lines such as QH;gen;gi;out = QH;nd;zi,mi / ηH;gen;gi,mi if we were modeling generation efficiency. However, for an Ideal Loads approach, we only store the occupant comfort setpoints and later pass them to EnergyPlus, skipping the explicit generation formula in code.

When the main script runs, it merges the user overrides into the default lookup, picks final numeric setpoints by either taking the midpoint of a (min_val, max_val) range or a random value depending on strategy, then passes these final numbers to the IDF creation routines. The schedule creation step is performed by add_HVAC_Ideal_to_all_zones, which writes a simplistic day schedule from 07:00–19:00 and a night schedule otherwise, referencing the final numeric setpoints from hvac_params. We could, if we wish, incorporate schedule_details from the lookup to build a more advanced SCHEDULE:COMPACT object that truly starts at, for example, day_start=06:00 or ends at day_end=22:00.

In summary, our integration of NTA 8800 in this script primarily concerns the occupant-facing values: day and night setpoints, setpoint corrections for different emission and control systems, typical infiltration assumptions by age, and occupant schedules by building type. By codifying these references in a Python dictionary, we can systematically assign them across a portfolio of buildings. Then, if the user wishes, we can refine or override these assumptions from an Excel file, storing the new numeric ranges or schedule details and merging them back into the primary lookup. Although we do not replicate the entirety of NTA 8800’s monthly calculation (with boiler, heat pump, or distribution losses) in code, we ensure that each building’s basic comfort parameters and daily schedules remain consistent with the standard. This provides a powerful mechanism to run simulations in EnergyPlus that approximate the occupant comfort assumptions of NTA 8800 while still allowing advanced load modeling.











